\documentclass{article}
\usepackage{amsmath}
\usepackage{amssymb}
\usepackage{amsthm}
\newtheorem{theorem}{Theorem}[section]
\newtheorem{lemma}[theorem]{Lemma}
\newtheorem{corollary}[theorem]{Corollary}
\renewcommand*{\proofname}{Proof}
\theoremstyle{definition}
\newtheorem{definition}{Definition}[section]
\theoremstyle{remark}
\newtheorem*{remark}{Remark}
\theoremstyle{example}
\newtheorem{example}{Example}[section]
\DeclareMathOperator*{\argmax}{arg\,max}
\usepackage{fullpage}
\usepackage{outlines}
\allowdisplaybreaks

\newcommand{\tif}{\text{if }}

\title{Measure Theory}
\author{Lysandre Terrisse}

\begin{document}
\maketitle

%https://www.math.lsu.edu/~sengupta/7312s02/sigmaalg.pdf
%https://moodle.umontpellier.fr/pluginfile.php/2373208/mod_resource/content/12/HAX503X-poly.pdf

\section{Measurable space and $\sigma$-algebra}

\begin{definition}[Measurable space and $\sigma$-algebra]
    Let $X$ be a set. A \textit{measurable space} is a couple $(X, \mathcal{A})$ where $\mathcal{A}$ is a \textit{$\sigma$-algebra} of $X$. That is, $\mathcal{A}$ is a set of subsets of $X$, called \textit{measurable sets}, such that:
    \begin{itemize}
        \item $X$ is a measurable set:
            $$X \in \mathcal{A}$$
        \item Complements of measurable sets are measurable:
            $$\forall A \in \mathcal{A}, A^c \in \mathcal{A}$$
        \item Countable unions of measurable sets are measurable: For all countable family $(A_i)_{i \in \mathbb{N}}$ of measurable sets:
            $$\bigcup_{i \in \mathbb{N}} A_i \in \mathcal{A}$$
    \end{itemize}
\end{definition}

\begin{theorem}[Countable intersections of measurable sets are measurable]
    Let $(X, \mathcal{A})$ be a measurable space, and let $(A_i)_{i \in \mathbb{N}}$ be a countable family over measurable sets. Then:
            $$\bigcap_{i \in \mathbb{N}} A_i \in \mathcal{A}$$
\end{theorem}

\begin{proof}
    As $\bigcup_{i \in \mathbb{N}} A_i \in \mathcal{A}$, then $\left(\bigcup_{i \in \mathbb{N}} A_i\right)^c = \bigcap_{i \in \mathbb{N}} A_i \in \mathcal{A}$.
\end{proof}

\begin{definition}[Trivial and full $\sigma$-algebra]
    Let $X$ be a set. The \textit{trivial $\sigma$-algebra} is $\{\varnothing, X\}$, and the \textit{full $\sigma$-algebra} is $\mathcal{P}(X)$. They indeed are $\sigma$-algebras, since $X$ belongs to them, and that they are stable over complements and countable unions.
\end{definition}

\begin{theorem}[$\sigma$-algebra are between the trivial and full $\sigma$-algebras]
    For all measurable space $(X, \mathcal{A})$, we have:
        $$\{\varnothing, X\} \subseteq \mathcal{A} \subseteq \mathcal{P}(X)$$
\end{theorem}

\begin{proof}
    Since $X \in \mathcal{A}$, and that $\mathcal{A}$ is closed under complement, then $X^c = X \setminus X = \varnothing \in \mathcal{A}$. Therefore, $\{\varnothing, X\} \subseteq \mathcal{A}$. Furthermore, from the definition of $\sigma$-algebra, $\mathcal{A} \subseteq \mathcal{P}(X)$. 
\end{proof}

\begin{remark}
    This theorem implies that the trivial and the full $\sigma$-algebras are respectively the smallest and biggest $\sigma$-algebras. By \textit{smallest}, we mean that all other $\sigma$-algebras of $X$ contain it, and by \textit{biggest}, we mean that all other $\sigma$-algebras of $X$ are contained in it.
\end{remark}

\begin{theorem}[Intersections of $\sigma$-algebras are $\sigma$-algebras]
    Let $X$ be a set, and let $(\mathcal{A}_i)_{i \in I}$ be a (potentially uncountable) family of $\sigma$-algebras of $X$. Then $\bigcap_{i \in I} \mathcal{A}_i$ is a $\sigma$-algebra.
\end{theorem}

\begin{proof}~
    \begin{itemize}
        \item As we have $X \in \mathcal{A}_i$ for all $i \in I$, then we have $X \in \bigcap_{i \in I} \mathcal{A}_i$.
        \item If $A \in \bigcap_{i \in I} \mathcal{A}_i$, then $A \in \mathcal{A}_i$ for all $i \in I$, and therefore $A^c \in \mathcal{A}_i$ for all $i \in I$, meaning that $A^c \in \bigcap_{i \in I} \mathcal{A}_i$.
        \item Let $(A_j)_{j \in J}$ be a countable family over $\bigcap_{i \in I} \mathcal{A}_i$. This family is also over $\mathcal{A}_i$ for all $i \in I$. Therefore, $\bigcup_{j \in J} A_j \in \mathcal{A}_i$ for all $i \in I$, meaning that $\bigcup_{j \in J} A_j \in \bigcap_{i \in I} \mathcal{A}_i$.
    \end{itemize}
\end{proof}

\begin{remark}
    Unions of $\sigma$-algebras are not necessarily $\sigma$-algebras.
\end{remark}

\begin{definition}[Generated $\sigma$-algebra]
    Let $X$ be a set, and let $Y \subseteq \mathcal{P}(X)$. The $\sigma$-algebra of $X$ generated by $Y$, denoted $\sigma(Y)$, is the smallest $\sigma$-algebra of $X$ containing all the sets of $Y$. That is, $\sigma(Y)$ is defined as the only set such that:
    \begin{itemize}
        \item $\sigma(Y)$ is a $\sigma$-algebra of $X$
        \item $Y \subseteq \sigma(Y)$
        \item For all $\sigma$-algebra $\mathcal{A}$ of $X$ such that $Y \subseteq \mathcal{A}$, we have $\sigma(Y) \subseteq \mathcal{A}$.
    \end{itemize}
    The uniqueness of such $\sigma$-algebra can be proven directly: for any two sets respecting these properties, they would from the third property include each other, and would therefore be equal. The existence such $\sigma$-algebra is proven in the next theorem.
\end{definition}

\begin{theorem}[Characteristic property of generated $\sigma$-algebras]
    Let $X$ be a set, let $Y \subseteq \mathcal{P}(X)$, and let $Z = \{\mathcal{A} \mid \text{$Y \subseteq \mathcal{A}$ and $\mathcal{A}$ is a $\sigma$-algebra of $X$}\}$. Then we have:
        $$\sigma(Y) = \bigcap_{\mathcal{A} \in Z} \mathcal{A}$$
\end{theorem}

\begin{proof}
    Firstly, $\bigcap_{\mathcal{A} \in Z} \mathcal{A}$ is an intersection of $\sigma$-algebras of $X$, and is therefore a $\sigma$-algebra of $X$. Secondly, as $Y \subseteq \mathcal{A}$ for all $\mathcal{A} \in Z$, we have that $Y \subseteq \bigcap_{\mathcal{A} \in Z} \mathcal{A}$. Thirdly, $\bigcap_{\mathcal{A} \in Z} \mathcal{A} \subseteq \mathcal{A}'$ for all $\mathcal{A}' \in Z$. Therefore, $\bigcap_{\mathcal{A} \in Z} \mathcal{A}$ respects all the three properties of the definition of $\sigma(Y)$, meaning that $\sigma(Y) = \bigcap_{\mathcal{A} \in Z} \mathcal{A}$.
\end{proof}

\begin{definition}[Borel $\sigma$-algebra]
    Let $(E, T)$ be a \fbox{topological space}. We define the Borel $\sigma$-algebra of $(E, T)$, denoted $\mathcal{B}(E, T)$, as the $\sigma$-algebra generated by $T$. That is, $\mathcal{B}(E, T)$ is the smallest $\sigma$-algebra of $E$ which contains all the open sets of $E$. When there is no confusion about which topology is used, we will write $\mathcal{B}(E)$ instead of $\mathcal{B}(E, T)$.
\end{definition}

\section{Measure spaces and measures}

\begin{definition}[Measure spaces and measures]
    Let $(X, \mathcal{A})$ be a measurable space. A measure space is a tuple $(X, \mathcal{A}, \mu)$, where $\mu$ is a \textit{measure} over $(X, \mathcal{A})$. That is, $\mu$ is a function of type $\mathcal{A} \rightarrow \mathbb{R}_+ \cup \{\infty\}$ such that:
    \begin{itemize}
        \item $\mu(\varnothing) = 0$
        \item For all countable family $(A_i)_{i \in \mathbb{N}}$ of disjoint measurable sets of $\mathcal{A}$, we have:
            $$\mu(\bigcup_{i \in \mathbb{N}} A_i) = \sum_{i \in \mathbb{N}} \mu(A_i)$$
    \end{itemize}
\end{definition}

\begin{remark}
    Do not be mistaken between measurable spaces and measure spaces. A measure space is a measurable space with a measure.
\end{remark}

\begin{definition}[Negligible set]
    Let $(X, \mathcal{A}, \mu)$ be a measure space. For all $A \in \mathcal{A}$, we say that $A$ is \textit{negligible} when $\mu(A) = 0$. For instance, the empty set is always negligible.
\end{definition}

\begin{definition}[Total mass]
    Let $(X, \mathcal{A}, \mu)$ be a measure space. The total mass of $\mu$ is $\mu(X)$.
\end{definition}

\begin{definition}[Increasing and decreasing families of sets]
    Let $(A_i)_{i \in I}$ be a family of sets. We say that $(A_i)_{i \in I}$ is increasing when:
         $$\forall i \in I, A_i \subseteq A_{i+1}$$
    Similarly, we say that $(A_i)_{i \in I}$ is decreasing when:
         $$\forall i \in I, A_i \supseteq A_{i+1}$$
\end{definition}

\begin{definition}[Finiteness and $\sigma$-finiteness]
    We say that $\mu$ is finite when its total mass isn't $\infty$, and we say that $\mu$ is $\sigma$-finite when there exists a countable family $(A_i)_{i \in \mathbb{N}}$ such that:
    \begin{itemize}
        \item $(A_i)_{i \in \mathbb{N}}$ is increasing:
            $$\forall i \in \mathbb{N}, A_i \subseteq A_{i+1}$$
        \item $(A_i)_{i \in \mathbb{N}}$ converges towards $X$:
            $$X = \bigcup_{i \in \mathbb{N}} A_i$$
        \item $(A_i)_{i \in \mathbb{N}}$ has no element of infinite mass:
            $$\forall i \in \mathbb{N}, \mu(A_i) \neq \infty$$
    \end{itemize}
\end{definition}

\begin{remark}
    Finiteness implies $\sigma$-finiteness, but $\sigma$-finiteness doesn't necessarily imply finiteness.
\end{remark}

\begin{definition}[Dirac measure]
    Let $(X, \mathcal{A})$ be a measurable space, and let $x \in X$. The \textit{Dirac measure at $x$} is defined as:
        $$\delta_x : \mathcal{A} \rightarrow \mathbb{R}_+ \cup \{\infty\}$$
        $$A \mapsto \begin{cases} 1 & \tif x \in A\\ 0 & \tif x \notin A\end{cases}$$
    Let's prove that the Dirac measure is indeed a measure. Firstly, as $x \notin \varnothing$, then $\delta_x(\varnothing) = 0$. Secondly, let $(A_i)_{i \in \mathbb{N}}$ be a countable family of disjoint measurable sets. We have two cases:
    \begin{itemize}
        \item If $x \in \bigcup_{i \in \mathbb{N}} A_i$, then, as the $A_i$ are disjoint, there exists a unique $j \in \mathbb{N}$ such that $x \in A_j$, and for all other $i \in \mathbb{N} \setminus \{j\}$, $x \notin A_i$. Therefore:
            $$\sum_{i \in \mathbb{N}} \delta_x(A_i) = \delta_x(A_j) + \sum_{i \in \mathbb{N} \setminus \{j\}} \delta_x(A_i) = 1 + \sum_{i \in \mathbb{N}} 0 = 1 = \delta_x(\bigcup_{i \in \mathbb{N}} A_i)$$
        \item If $x \notin \bigcup_{i \in \mathbb{N}} A_i$, then for all $i \in \mathbb{N}$, $x \notin A_i$, and therefore:
        $$\sum_{i \in \mathbb{N}} \delta_x(A_i) = \sum_{i \in \mathbb{N}} 0 = 0 = \delta_x(\bigcup_{i \in \mathbb{N}} A_i)$$
    \end{itemize}
\end{definition}

\begin{theorem}[Positive linear combinations of measures are measures] \label{thm:pos_lin_comb_measures}
    Let $(X, \mathcal{A})$ be a measurable space, let $(\mu_i)_{i \in \mathbb{N}}$ be a countable family of measures over $(X, \mathcal{A})$, and let $(a_i)_{i \in \mathbb{N}}$ be a countable family of elements of $\mathbb{R}_+$. Then $\mu = \sum_{i \in \mathbb{N}} a_i \mu_i$ is a measure over $(X, \mathcal{A})$.
\end{theorem}

\begin{proof}~
    \begin{itemize}
        \item $\left(\sum_{i \in \mathbb{N}} a_i \mu_i\right)(\varnothing) = \sum_{i \in \mathbb{N}} a_i \mu_i(\varnothing) = \sum_{i \in \mathbb{N}} 0 = 0$
        \item Let $(A_j)_{j \in \mathbb{N}}$ be a countable family of distinct measurable sets of $\mathcal{A}$. Then we have:
        \begin{align*}
            \mu(\bigcup_{j \in \mathbb{N}} A_j) &= \left(\sum_{i \in \mathbb{N}} a_i \mu_i\right)(\bigcup_{j \in \mathbb{N}} A_j)\\
            &= \sum_{i \in \mathbb{N}} a_i \mu_i(\bigcup_{j \in \mathbb{N}} A_j)\\
            &= \sum_{i \in \mathbb{N}} a_i \sum_{j \in \mathbb{N}} \mu_i(A_j)\\
            &= \sum_{i \in \mathbb{N}} \sum_{j \in \mathbb{N}} a_i \mu_i(A_j)\\
            &= \sum_{j \in \mathbb{N}} \sum_{i \in \mathbb{N}} a_i \mu_i(A_j)\\
            &= \sum_{j \in \mathbb{N}} \left(\sum_{i \in \mathbb{N}} a_i \mu_i\right)(A_j)\\
            &= \sum_{j \in \mathbb{N}} \mu(A_j)
        \end{align*}
    \end{itemize}
\end{proof}

\begin{definition}[Discrete measures]
    Let $(X, \mathcal{A})$ be a measurable space. A \textit{discrete measure} is a positive linear combination of Dirac measures (which from theorem (\ref{thm:pos_lin_comb_measures}) is a measure). That is, let $(a_i)_{i \in \mathbb{N}}$ be a countable family of elements of $\mathbb{R}_+$, and let $(x_i)_{i \in \mathbb{N}}$ be a countable family of elements of $X$. Then $\sum_{i \in \mathbb{N}} a_i \delta_{x_i}$ is a discrete measure, where $\delta_{x_i}$ represents the Dirac measure at $x_i$.
\end{definition}

\begin{definition}[Counting measure]
    Let $(X, \mathcal{A})$ be a measurable space. The counting measure is defined as:
        $$\chi(A) = \begin{cases} Card(A) & \tif A \text{ is countable}\\ \infty & \tif A \text{ is uncountable}\end{cases} = \begin{cases}Card(A) & \tif A \text{ is finite}\\ \infty & \tif A \text{ is infinite}\end{cases}$$
    If $X$ is countable, then $\chi$ is a discrete measure since for all $A \in \mathcal{A}$, we have $\chi(A) = Card(A) = \sum_{x \in A} 1 = \sum_{x \in A} \delta_x(A) = \left(\sum_{x \in A} \delta_x\right)(A)$.
\end{definition}

%\begin{theorem}
%    Let $(X, \mathcal{A}, \mu)$ be a measure space. Then the following four properties are true:
%    \begin{itemize}
%        \item Monotonicity:
%            $$\forall A, B \in \mathcal{A}, A \subseteq B \implies \mu{A} \leq \mu{B}$$
%        \item Subadditivity: Let $(A_i)_{i \in \mathbb{N}}$ be a countable family of (not necessarily distinct) measurable sets of $\mathcal{A}$. Then we have:
%            $$\mu(\bigcup_{i \in \mathbb{N}} A_i) \leq \sum_{i \in \mathbb{N}} \mu(A_i)$$
%        \item Let $(A_i)_{i \in \mathbb{N}}$ be an increasing countable family of measurable sets of $\mathcal{A}$. Then we have:
%            $$\mu(\sum_{i \in \mathbb{N}} A_i) = \lim_{i \rightarrow \infty} \mu(A_i)$$
%        \item Same but for decreasing sequences
%    \end{itemize}
    %https://proofwiki.org/wiki/Measure_of_Limit_of_Increasing_Sequence_of_Measurable_Sets
%\end{theorem}

\begin{definition}[Rectangular cuboid]
	We say that $C \subseteq \mathbb{R}^n$ is a rectangular cuboid when it is of the form $C = [a_1, b_1] \times \dots \times [a_n, b_n]$. When introducing a rectangular cuboid of this form, we will always assume that $a_i \leq b_i$ for all $i \in \{1, \dots, n\}$.
\end{definition}

\begin{theorem}[Rectangular cuboids are borel sets]
	For all rectangular cuboid $C = [a_1, b_1] \times \dots \times [a_n, b_n]$, we have that $C \in \mathcal{B}(\mathbb{R}^n)$.
\end{theorem}

\begin{proof}
	\fbox{Requires the notion of all norms being equivalent.}
\end{proof}

\begin{definition}[Lebesgue Measure]
    Let's consider the measurable space $(\mathbb{R}^n, \mathcal{B}(\mathbb{R}^n))$. We define the Lebesgue measure over $(\mathbb{R}^n, \mathcal{B}(\mathbb{R}^n))$, denoted $\lambda$, as the only measure such that, for all rectangular cuboid $C = [a_1, b_1] \times \dots \times [a_n, b_n] \in \mathcal{B}(\mathbb{R}^n)$, we have:
        $$\lambda(C) = \prod_{i=1}^n (b_i - a_i)$$
    Let's now prove the existence and uniqueness of the Lebesgue measure. %TODO
\end{definition}




\end{document}
