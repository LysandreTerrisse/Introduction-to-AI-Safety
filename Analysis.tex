\documentclass{article}
\usepackage{amsmath}
\usepackage{amssymb}
\usepackage{amsthm}
\newtheorem{theorem}{Theorem}[section]
\newtheorem{lemma}[theorem]{Lemma}
\newtheorem{corollary}[theorem]{Corollary}
\renewcommand*{\proofname}{Proof}
\theoremstyle{definition}
\newtheorem{definition}{Definition}[section]
\theoremstyle{remark}
\newtheorem*{remark}{Remark}
\theoremstyle{example}
\newtheorem{example}{Example}[section]
\theoremstyle{notation}
\newtheorem{notation}{Notation}[section]
\DeclareMathOperator*{\argmax}{arg\,max}
\usepackage{fullpage}
\usepackage{accents}
\usepackage{outlines}
\allowdisplaybreaks

\newcommand{\inter}[1]{{{#1}^\circ}}
\newcommand{\adher}[1]{{\overline{#1}}}
\newcommand{\bound}[1]{{\partial #1}}
\newcommand{\exter}[1]{{{#1}^e}}
\newcommand{\limit}[1]{{{#1}'}}
\newcommand{\isol}[1]{{{#1}^i}}

\newcommand{\closed}{\mathcal{C}}
\newcommand{\lr}[1]{\left(#1\right)}
\newcommand{\tand}{\text{ and }}

\title{Analysis}
\author{Lysandre Terrisse}

\begin{document}
\maketitle

\section{Topological space}

%https://www.math.ucdavis.edu/~hunter/book/ch4.pdf
%https://fr.wikipedia.org/wiki/Espace_topologique
%http://math.soimeme.org/~arunram/Preprints/181212Cntfcns.pdf
\begin{definition}[Topological space and topology]
		$(E, T)$ is a topological space when $E$ is a set and $T$ is a \textit{topology} over $E$. That is, $T$ is a set of subsets of $E$, called open sets, such that:
		\begin{itemize}
				\item $\varnothing$ and $E$ are open: $\varnothing, E \in T$
				\item Unions (potentially infinite) of opens are open: Let $S \subseteq T$ be a set of opens. Then:
						$$\bigcup_{O \in S} O \in T$$
				\item Finite intersections of opens are open: Let $S = \{O_1, \dots, O_n\} \subseteq T$ be a finite set of opens. Then:
						$$\bigcap_{O \in S} O \in T$$
		\end{itemize}
\end{definition}

\begin{example}[Trivial topology]
		For any set $E$, we call $\{\varnothing, E\}$ the trivial topology of $E$. It is the smallest topology of $E$ by inclusion, that is, this topology is included in every other topology of $E$.
\end{example}

\begin{example}[Discrete topology]
		For any set $E$, we call $\mathcal{P}(E)$ the discrete topology of $E$. It is the biggest topology of $E$ by inclusion, that is, this topology includes every other topology of $E$.
\end{example}

\begin{definition}[Closed set]
		Let $(E, T)$ be a topological space. A subset $C \subseteq E$ is said to be closed when its complementary is open. We call $\closed = \{C \subseteq E \mid C^c \in T\}$ the set of closed sets. Alternatively, we could say that a subset $C \subseteq E$ is closed when it is the complementary of an open (in that case, it is the complementary of the open $C^c$). As such, we have $\closed = \{O^c \mid O \in T\}$.
\end{definition}

\begin{remark}
	Just as closed sets are sets whose complementary is open, open sets are sets whose complementary is closed, and therefore $T = \{O \subseteq E \mid O^c \in \closed\}$. Similarly, just as closed sets are the complementaries of open sets, open sets are the complementaries of closed sets, and therefore $T = \{C^c \mid C \in \closed\}$.
\end{remark}

\begin{definition}[Clopen sets]
		Let $(E, T)$ be a topological space. We say that $X \subseteq E$ is clopen when it is both open and closed.
\end{definition}

\begin{theorem}[$E$ and $\varnothing$ are clopen]
		Let $(E, T)$ be a topological space. Then, $\varnothing$ and $E$ are clopen.
\end{theorem}

\begin{proof}
		From the definition of topological space, we have that $\varnothing$ and $E$ are open. But, as they are both the complementary of each other, then they are the complementary of an open. They therefore are also closed.
\end{proof}

\begin{theorem}[Intersection of closed sets is closed]
		Let $(E, T)$ be a topological space, and let $S \subseteq \closed$ be a set of closed sets. Then $\bigcap_{C \in S} C$ is closed.
\end{theorem}

\begin{proof}
		To prove that $\bigcap_{C \in S} C$ is closed, we will prove that its complementary its open. From De Morgan's laws, we have:
				$$(\bigcap_{C \in S} C)^c = \bigcup_{C \in S} C^c$$
		And as all the $C \in S$ are closed, then the $C^c$ are open. We therefore have a union of opens, which is open.
\end{proof}

\begin{theorem}[Finite union of closed sets is closed]
		Let $(E, T)$ be a topological space, and let $S = \{C_1, \dots, C_n\} \subseteq \closed$ be a finite set of closed sets. Then $\bigcup_{C \in S} C$ is closed.
\end{theorem}

\begin{proof}
		From De Morgan's laws, the complementary of $\bigcup_{C \in S} C$ is:
				$$(\bigcup_{C \in S} C)^c = \bigcap_{C \in S} C^c$$
		which is a finite intersection of open, and is therefore open. Therefore, $\bigcup_{C \in S} C$ is closed.
\end{proof}

\section{Neighbourhood}

\begin{definition}[Neighbourhood of a set]
		Let $(E, T)$ be a topological space, and let $X \subseteq E$. A neighbourhood of $X$ is a subset $N$ that includes an open set which includes $X$. That is, $N$ is a neighbourhood of $X$ when:
				$$\exists O \in T, X \subseteq O \subseteq N$$
		We write $\mathcal{N}_E(X)$ for the set of all neighbourhoods of $X$ according to the topological space $(E, T)$. When there is no confusion about which topological space is considered, we just write $\mathcal{N}(X)$.
\end{definition}

\begin{definition}[Neighbourhood of a point]
		Let $(E, T)$ be a topological space, and let $x \in E$. $N \subseteq E$ is said to be a neighbourhood of $x$ when it is a neighbourhood of the singleton $\{x\}$. Equivalently, it is when there exists an open $O \in T$ such that $x \in O$ and $O \subseteq N$. We also use the notation $\mathcal{N}(x)$ for the neighbourhood of $x$.
\end{definition}

\begin{theorem}[All sets have a clopen neighbourhood] \label{thm:neighbourhood-existence}
		Let $(E, T)$ be a topological space, and let $X \subseteq E$. Then, $X$ has a clopen neighbourhood.
\end{theorem}

\begin{proof}
		This is trivial as $E$ is a clopen and is a neighbourhood of $X$ (since $X \subseteq E \subseteq E$ holds).
\end{proof}

\begin{theorem}[A set is open iff it is a neighbourhood of itself] \label{thm:open-eq-own-neighbourhood}
		Let $(E, T)$ be a topological space, and let $X \subseteq E$. $X$ is open if and only if it is a neighbourhood of itself.
\end{theorem}

\begin{proof}~
	\begin{itemize}
		\item Suppose that $X$ is open. As $X \subseteq X \subseteq X$ and that $X$ is open, then $X$ is a neighbourhood of itself.
		\item Suppose that $X$ is a neighbourhood of itself. Then, there exists an open $O \in T$ such that $X \subseteq O \subseteq X$, and therefore $X = O$, proving that $X$ is open.
	\end{itemize}
\end{proof}

\begin{theorem}[A set is open iff it is a neighbourhood of all its subsets] \label{thm:open-eq-neighbourhood}
		Let $(E, T)$ be a topological space, and let $X \subseteq E$. Then $X$ is open if and only if it is a neighbourhood of all its subsets.
\end{theorem}

\begin{proof}~
	\begin{itemize}
		\item Suppose that $X$ is open, and let $X' \subseteq X$. As $X' \subseteq X \subseteq X$, and that $X$ is open, then $X$ indeed is a neighbourhood of $X'$.
		\item Suppose that $X$ is a neighbourhood of all of its subsets. Then it is also a neighbourhood of itself, and from theorem (\ref{thm:open-eq-own-neighbourhood}), $X$ is open.
	\end{itemize}
\end{proof}

\begin{theorem}[A set is open iff it is a neighbourhood of all its points] \label{thm:open-eq-point}
		Let $(E, T)$ be a topological space, and let $X \subseteq E$. Then $X$ is open if and only if it is a neighbourhood of all its points.
\end{theorem}

\begin{proof}~
		\begin{itemize}
				\item Suppose that $X$ is open, and let $x \in X$. Then, from theorem (\ref{thm:open-eq-neighbourhood}), it is a neighbourhood of $\{x\}$ and therefore of $x$.
				\item Suppose that $X$ is a neighbourhood of all its points. Then, for each $x \in X$, there exists $O_x$ such that $x \in O_x$ and $O_x \subseteq X$. As for all $x \in X$, $O_x \subseteq X$, then the union $\bigcup_{x \in X} O_x$ also belongs to $X$. Furthermore, as for all $x \in X$, $x \in O_x$, then $x \in \bigcup_{x \in X} O_X$, meaning that $X \subseteq \bigcup_{x \in X} O_X$. Thus, $X = \bigcup_{x \in X} O_x$, which is open because it is an union of opens.
		\end{itemize}
\end{proof}

\begin{theorem}[An open containing a point is a neighbourhood of it] \label{thm:open-contain-point}
		Let $(E, T)$ be a topological space, let $O \in T$, and let $x \in O$. Then, $O \in \mathcal{N}(x)$.
\end{theorem}

\begin{proof}
		As $x \in O \subseteq O$, and that $O \in T$, then $O \in \mathcal{N}(x)$.
\end{proof}



\section{Structures formed by neighbourhoods}

In the next sections, $(E, T)$ will correspond to a topological space, and $X$ will correspond to a subset of $E$.

\begin{definition}[Adhrerent point and closure]
		$x \in E$ is said to be an adherent point of $X$ when every neighbourhood of $x$ intersect $X$:
				$$\forall N \in \mathcal{N}(x), X \cap N \neq \varnothing$$
		The closure $\adher{X}$ of $X$ is the set of all adherent points of $X$.
\end{definition}

\begin{definition}[Interior point and interior]
		$x \in E$ is said to be an interior point of $X$ when $X$ contains a neighbourhood of $x$:
				$$\exists N \in \mathcal{N}(X), N \subseteq X$$
		The interior $\inter{X}$ of $X$ is the set of all interior points of $X$.
\end{definition}

\begin{definition}[Boundary point and boundary]
		$x \in E$ is said to be a boundary point of $X$ when every neighbourhood of $x$ contains at least one point in $X$ and one point not in $X$:
				$$\forall N \in \mathcal{N}(X), X \cap N \neq \varnothing \tand X^c \cap N \neq \varnothing$$
		The boundary $\bound{X}$ of $X$ is the set of all boundary points of $X$.
\end{definition}

\begin{definition}[Exterior point and exterior]
		$x \in E$ is said to be an exterior point of $X$ when $x$ has a neighbourhood which doesn't intersect with $X$:
				$$\exists N \in \mathcal{N}(x), X \cap N = \varnothing$$
		The exterior $\exter{X}$ of $X$ is the set of all exterior points of $X$.
\end{definition}

\begin{definition}[Limit point and derived set]
		$x \in E$ is said to be a limit point of $X$ when every neighbourhood of $x$, after removing $x$ from it, intersect $X$:
				$$\forall N \in \mathcal{N}(X), X \cap (N \setminus \{x\}) \neq \varnothing$$
		The derived set $\limit{X}$ of $X$ is the set of all limit points of $X$.
\end{definition}

\begin{definition}[Isolated point and set of isolated points]
		$x \in E$ is said to be an isolated point of $X$ when $x$ has a neighbourhood whose intersection with $X$ is the singleton $\{x\}$:
				$$\exists N \in \mathcal{N}(X), X \cap N = \{x\}$$
		We note $\isol{X}$ for the set of all isolated points of $X$.
\end{definition}

As can be expected, these different structures have many common properties, some of which are resumed in the following table, where the intersection between the $i$-th row and $j$-th column tells how to define the $i$-th structure from the $j$-th structure when possible. Although each property can be easily proven, we will not discuss these proofs here, except for the duality between closure and interior.

\begin{center}
		\begin{tabular}{|c|c|c|c|c|c|c|}
				\hline
						    & $\adher{X}$                               & $\inter{X}$                         & $\bound{X}$                                                                  & $\exter{X}$                         & $\limit{X}$                                      & $\isol{X}$\\
				\hline
				$\adher{X}$ & $\adher{X}$                               & $\inter{X^c}^c$                     & $X \cup \bound{X}$                                                           & $\exter{X}^c$                       & $X \cup \limit{X}$                               &           \\
				$\inter{X}$ & $\adher{X^c}^c$                           & $\inter{X}$                         & $X \setminus \bound{X}$                                                      & $\exter{X^c}$                       & $X \setminus \limit{X^c}$                        &           \\
				$\bound{X}$ & $\adher{X} \cap \adher{X^c}$              & $\inter{X^c}^c \cap \inter{X}^c$    & $\bound{X}$                                                                  & $\exter{X}^c \cap \exter{X^c}^c$    & $(X \cup \limit{X}) \cap (X^c \cup \limit{X^c})$ &           \\
				$\exter{X}$ & $\adher{X}^c$                             & $\inter{X^c}$                       & $(X \cup \bound{X})^c$                                                       & $\exter{X}$                         & $(X \cup \limit{X})^c$                           &           \\
				$\limit{X}$ & $\adher{\adher{X^c}^c}$                   & $\inter{\inter{X}^c}^c$             & $(X \setminus \bound{X}) \cup \bound{(X \setminus \bound{X})}$               & $\exter{\exter{X^c}}^c$             & $\limit{X}$                                      &           \\
				$\isol{X}$  & $X \setminus \overline{\overline{X^c}^c}$ & $X \setminus \inter{\inter{X}^c}^c$ & $X \setminus ((X \setminus \bound{X}) \cup \bound{(X \setminus \bound{X}))}$ & $X \setminus \exter{\exter{X^c}}^c$ & $X \setminus \limit{X}$                          & $\isol{X}$\\
				\hline
		\end{tabular}
\end{center}

\begin{theorem}[Closure and interior are dual]
		We have:
				$$\adher{X} = \inter{X^c}^c$$
				$$\inter{X} = \adher{X^c}^c$$	
\end{theorem}

\begin{proof}
		Let's prove $\adher{X} = \inter{X^c}^c$:
		\begin{itemize}
				\item Let $x \in \adher{X}$. All of $x$'s neighbourhood therefore intersect $X$. There therefore is no neighbourhood of $x$ which don't intersect $X$. There therefore is no neighbourhood of $x$ which are contained in $X^c$. Therefore, $x \notin \inter{X^c}$, which means that $x \in \inter{X^c}^c$ and thus $\adher{X} \subseteq \inter{X^c}^c$.
				\item Let $x \in \inter{X^c}^c$. As $x \notin \inter{X^c}$, then there is no neighbourhood of $x$ included in $X^c$, and therefore every neighbourhood of $x$ must intersect $X$, implying that $x \in \adher{X}$ and thus $\inter{X^c}^c \subseteq \adher{X}$.
		\end{itemize}
		Let's now prove $\inter{X} = \adher{X^c}^c$:
		\begin{itemize}
				\item Let $x \in \inter{X}$. $X$ therefore contains a neighbourhood of $X$. Since this neighbourhood is contained in $X$, it cannot intersect $X^c$. Since $x$'s neighbourhoods therefore do not all intersect $X^c$, then $x \notin \adher{X^c}$, and therefore $x \in \adher{X^c}^c$, meaning that $\inter{X} \subseteq \adher{X^c}^c$.
				\item Let $x \in \adher{X^c}^c$. We therefore have that $x \notin \adher{X^c}$, which means that there is a neighbourhood of $x$ which doesn't intersect with $X^c$. This neighbourhood must therefore be contained in $X$, and we therefore have $x \in \inter{X}$, meaning that $\adher{X^c}^c \subseteq \inter{X}$.
		\end{itemize}
\end{proof}

\section{Characteristic properties of closure and interior}

\begin{notation}
		We note $\closed(X) = \{C \in \closed \mid X \subseteq C\}$ the set of all closed sets which include $X$, and we note $T(X) = \{O \in T \mid O \subseteq X\}$ the set of all open sets which are included in $X$.
\end{notation}

\begin{theorem}[The closure contains the set and the set contains the interior] \label{thm:interior-in-set-in-closure}
		We have:
				$$\inter{X} \subseteq X \subseteq \adher{X}$$
\end{theorem}

\begin{proof}
		For all $x \in \inter{X}$, $x$ has a neighbourhood $N \in \mathcal{N}(x)$ such that $x \in N \subseteq X$, and thus $\inter{X} \subseteq X$. For all $x \in X$, every neighbourhood of $x$ intersects $X$ at the point $x$, and therefore $x \in \adher{X}$, meaning that $X \subseteq \adher{X}$.
\end{proof}

\begin{theorem}[Closure is closed and interior is open] \label{thm:closure-closed-interior-open}
		$\adher{X}$ is closed for all $X \subseteq E$. As closure and interior are dual, this property is equivalent to saying that $\inter{X}$ is open for all $X \subseteq E$.
\end{theorem}

\begin{proof}
		Let's prove that $\inter{X}$ is open. Let $x \in \inter{X}$. From the definition of interior, there exists $N \in \mathcal{N}(x)$ such that $N \subseteq X$. As $N \in \mathcal{N}(x)$, there exists an open $O \in T$ such that $x \in O$ and $O \subseteq N$. For all $x' \in O$, we have that one of its neighbourhood (for instance $N$) is contained in $X$. Therefore, every point $x' \in O$ belongs to $\inter{X}$, meaning that $O \subseteq \inter{X}$. Therefore, $x \in O$ and $O \subseteq \inter{X}$, meaning that $\inter{X}$ is a neighbourhood of $x$. Thus, $\inter{X}$ is a neighbourhood of all its points, and from theorem (\ref{thm:open-eq-point}), it is open.
\end{proof}

\begin{theorem}[Closed sets are their own closure and open sets are their own interior]
		For all $C \in \closed$, $\adher{C} = C$. This implies that for all $O \in T$, $\inter{O} = O$ since we would have $\inter{O} = \adher{O^c}^c = {O^c}^c = O$.
\end{theorem}

\begin{proof}~
		As $C^c$ is open, it is a neighbourhood of all its points (from theorem (\ref{thm:open-contain-point})) which doesn't intersect $C$. Therefore, no point in $C^c$ is an adherent point of $C$. Therefore, all adherent points of $C$ belong to $C$, and therefore $\adher{C} \subseteq C$. Furthermore $C \subseteq \adher{C}$ from theorem (\ref{thm:interior-in-set-in-closure}).
\end{proof}

\begin{theorem}[Closure as an intersection]
		The closure $\adher{X}$ of $X$ is equal to the intersection of all closed sets that include $X$:
				$$\adher{X} = \bigcap_{C \in \closed(X)} C$$
\end{theorem}

\begin{proof}~
		\begin{itemize}
				\item Let $x \in \adher{X}$, and let $C \in \closed(X)$. Since every neighbourhood of $x$ intersect $X$, and that $X \subseteq C$, then every neighbourhood of $x$ intersects $C$, and thus $x \in \adher{C} = C$. Therefore, $\adher{X} \subseteq C$ for all $C \in \mathcal{C}{X}$, and thus $\adher{X} \subseteq \bigcap_{C \in \mathcal{C}(X)} C$
				\item As $X \subseteq \adher{X}$ and that $\adher{X} \in \closed$, then $\adher{X} \in \closed(X)$, and therefore $\bigcap_{C \in \closed(X)} C \subseteq \adher{X}$. 
		\end{itemize}
\end{proof}

\begin{theorem}[Interior as an union]
		The interior $\inter{X}$ of $X$ is equal to the union of all open sets that are included in $X$:
				$$\inter{X} = \bigcup_{O \in T(X)} O$$
\end{theorem}

\begin{proof}~
		\begin{itemize}
				\item As $\inter{X} \subseteq X$ and that $\inter{X} \in T$, then $\inter{X} \in T(X)$, and therefore $\inter{X} \subseteq \bigcup_{O \in T(X)} O$.
				\item Let $x \in \bigcup_{O \in T(X)} O$. Since $O \subseteq X$ for every $O \in T(X)$, then $\bigcup_{O \in T(X)} O \subseteq X$. Furthermore, as $\bigcup_{O \in T(X)} O$ is open, it is a neighbourhood of $x$ which is included in $X$, and therefore $x \in \inter{X}$, meaning that $\bigcup_{O \in T(X)} O \subseteq \inter{X}$. 
		\end{itemize}
\end{proof}


\begin{theorem}[Closure as the smallest closed set] \label{thm:closure-charac}
		Let $(E, T)$ be a topological space, and let $X \subseteq E$. $\overline{X}$ is the smallest closed set which includes $X$. That is, $\overline{X}$ is the only set such that:
		\begin{align*}
				\adher{X} &\in \closed & X &\subseteq \adher{X} & \forall C \in \closed(X), \adher{X} &\subseteq C
		\end{align*}
\end{theorem}

\begin{proof}~
		\begin{itemize}
				\item Let's firstly prove that $\overline{X}$ respects these three properties. The first two properties come from theorem (\ref{thm:closure-closed-interior-open}) and (\ref{thm:interior-in-set-in-closure}) respectively. For the third property, let $C \in \closed(X)$ and $x \in \adher{X}$. Since every neighbourhood of $x$ intersect $X$, and that $X \subseteq C$, then every neighbourhood of $x$ intersects $C$, and thus $x \in \adher{C} = C$, meaning that $\adher{C} \subseteq C$.
				\item We now need to prove that $\overline{X}$ is the only set respecting these three properties. Let's suppose there is another set $A$ such that $A \in \closed$, $X \subseteq A$, and $\forall C \in \closed(X), A \subseteq C$. Then, as $\bar{X} \in \closed$ (first property of $\bar{X}$) and $X \subseteq \bar{X}$ (second property of $\bar{X}$), then $\bar{X} \in \closed(X)$, and therefore, from the third property of $A$, we have that $A \subseteq \bar{X}$. Similarly, as $A \in \closed$ and $X \subseteq A$, then $A \in \closed(X)$, and therefore $\overline{X} \subseteq A$. Therefore, $A = \overline{X}$, which proves that $\overline{X}$ is the unique set respecting these three properties.
		\end{itemize}
\end{proof}

\begin{theorem}[Interior as the biggest open set] \label{thm:interior-charac}
		Let $(E, T)$ be a topological space, and let $X \subseteq E$. $\inter{X}$ is the biggest open set which is included in $X$. That is, $\inter{X}$ is the only set such that:
		\begin{align*}
				\inter{X} &\in T & \inter{X} &\subseteq X & \forall O \in T(X), O &\subseteq \inter{X}
		\end{align*}
\end{theorem}

\begin{proof}~
		\begin{itemize}
				\item As in the previous proof, the first two points come from theorem (\ref{thm:closure-closed-interior-open}) and (\ref{thm:interior-in-set-in-closure}) respectively. For the third point, let $O \in T(X)$, and let $x \in O$. As $O$ is open, it is a neighbourhood of $x$ which is included in $X$, and therefore $x \in \inter{X}$, meaning that $O \subseteq \inter{X}$.
				\item If there were another set $A$ with these properties, then since it would be open and would be included in $X$, it would be in $T(X)$, and therefore we would have $A \subseteq \overline{X}$. And furthermore, as $\overline{X} \in T(X)$, we would also have $\overline{X} \subseteq A$, and therefore these two sets would be the same.
		\end{itemize} 
\end{proof}

\begin{theorem}[Characteristic property of adherent point] \label{thm:adherent-carac}
		The definition of adherent points also works if we consider only the open neighbourhoods instead of every neighbourhood. That is, for all $x \in E$, $x \in \adher{X}$ if and only if:
				$$\forall O \in \mathcal{N}(x) \cap T, O \cap X \neq \varnothing$$
\end{theorem}

\begin{proof}
		Let's prove the implication and then the reciprocal.
		\begin{itemize}
				\item If $x \in \adher{X}$, then as all neighbourhoods of $x$ have a non-null intersection with $X$, then this is also the case for all open neighbourhoods of $X$.
				\item Let $x \in E$ such that all open neighbourhoods of $x$ intersect $X$. Then, for any neighbourhood $N$ of $x$, it contains an open $O$ containing $x$. This open $O$ is itself an open neighbourhood of $x$, which therefore intersects $X$. As $O \subseteq N$, $N$ must also intersect $X$, which proves that $x \in \adher{X}$.
		\end{itemize}
\end{proof}

\begin{theorem}[Characteristic property of interior point]
		For all $x \in E$, $x \in \inter{X}$ if and only if $X$ is a neighbourhood of $x$.
\end{theorem}

\begin{proof}
		Let's prove the equivalence:
		\begin{itemize}
				\item If $x \in \inter{X}$, then there exists a neighbourhood $N \in \mathcal{N}(x)$ of $x$ such that $N \subseteq X$. From the definition of neighbourhood, there exists an open $O \in T$ such that $x \in O \subseteq N$, and therefore we have $x \in O \subseteq X$, which means that $X$ is a neighbourhood of $x$.
				\item If $X$ is a neighbourhood of $x$, then there trivially exists a neighbourhood $N \in \mathcal{N}(x)$ of $x$ such that $N \subseteq X$, as we can take $N = X$, and thus $x \in \inter{X}$.
		\end{itemize}
\end{proof}

\section{Other properties of closure and interior}

\begin{theorem}[Finite union of closures is the closure of the union]
		Let $X, Y \subseteq E$. Then, we have:
				$$\overline{X} \cup \overline{Y} = \overline{X \cup Y}$$
\end{theorem}

\begin{proof}~
		\begin{itemize}
				\item As $\overline{X \cup Y}$ is closed and includes $X$ (since $X \subseteq X \cup Y \subseteq \overline{X \cup Y}$), then it includes $\overline{X}$. Similarly, as $\overline{X \cup Y}$ is closed and includes $Y$, it includes $\overline{Y}$. Therefore, $\overline{X} \cup \overline{Y} \subseteq \overline{X \cup Y}$.
				\item As $X \subseteq \overline{X} \subseteq \overline{X} \cup \overline{Y}$ and $Y \subseteq \overline{Y} \subseteq \overline{X} \cup \overline{Y}$, then $X \cup Y \subseteq \overline{X} \cup \overline{Y}$. Since $\overline{X} \cup \overline{Y}$ is closed (since it is a finite union of closed sets) and includes $X \cup Y$, then it includes $\overline{X \cup Y}$. Therefore, $\overline{X \cup Y} \subseteq \overline{X} \cup \overline{Y}$.
		\end{itemize}
\end{proof}

\begin{corollary}[Finite intersection of interiors is the interior of the intersection]
		Let $X, Y \subseteq E$. Then, we have:
				$$\inter{X} \cap \inter{Y} = \inter{(X \cap Y)}$$
\end{corollary}

\begin{proof}
		$$\inter{X} \cap \inter{Y} = \overline{X^c}^c \cap \overline{Y^c}^c = (\overline{X^c} \cup \overline{Y^c})^c = \overline{X^c \cup Y^c}^c = \overline{(X \cap Y)^c}^c = \inter{(X \cap Y)}$$
\end{proof}

\begin{remark}
		The finite intersection of closures is not necessarily the closure of the intersection, and the finite union of interiors is not necessarily the interior of the union.
\end{remark}

\section{Limit}

\begin{notation}[Image of a set]
		Let $X$ and $Y$ be sets, let $f : X \rightarrow Y$, and let $A \subseteq X$. We write $f(A)$ for $\{f(a) \mid a \in A\}$. This notation isn't problematic, unless the set $A$ is itself in $X$, in which case we cannot know if $f(A)$ just means the image of $A$ or the set of images of all points of $A$.
\end{notation}

\begin{definition}[Limit]
		Let $(X, T_X)$ and $(Y, T_Y)$ be two topological spaces, and let $A \subseteq E$. We say that the limit of $f : A \rightarrow F$ at point $a \in \overline{A}$ is $l \in F$ when:
				$$\forall N \in \mathcal{N}_Y(y), \exists M \in \mathcal{N}_X(a), f(M \cap A) \subseteq N$$

		Let $(X, T_X)$ and $(Y, T_Y)$ be topological spaces. We say that the limit of $f : X \rightarrow Y$, at point $a \in X$ is $y \in Y$ when:
				$$\forall N \in \mathcal{N}_Y(y), \exists M \in \mathcal{N}_X(a), f(M) \subseteq N$$

\end{definition}



\end{document}
