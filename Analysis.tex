\documentclass{article}
\usepackage{amsmath}
\usepackage{amssymb}
\usepackage{amsthm}
\newtheorem{theorem}{Theorem}[section]
\newtheorem{lemma}[theorem]{Lemma}
\newtheorem{corollary}[theorem]{Corollary}
\renewcommand*{\proofname}{Proof}
\theoremstyle{definition}
\newtheorem{definition}{Definition}[section]
\theoremstyle{remark}
\newtheorem*{remark}{Remark}
\theoremstyle{example}
\newtheorem{example}{Example}[section]
\theoremstyle{notation}
\newtheorem{notation}{Notation}[section]
\DeclareMathOperator*{\argmax}{arg\,max}
\usepackage{fullpage}
\usepackage{accents}
\usepackage{outlines}
\allowdisplaybreaks

\newcommand{\inter}[1]{{{#1}^\circ}}
\newcommand{\closed}{\mathcal{C}}
\newcommand{\lr}[1]{\left(#1\right)}
\newcommand{\tand}{\text{ and }}

\title{Analysis}
\author{Lysandre Terrisse}

\begin{document}
\maketitle

\section{Topological space}

\subsection{Opens and closed sets}

%https://www.math.ucdavis.edu/~hunter/book/ch4.pdf
%https://fr.wikipedia.org/wiki/Espace_topologique
%http://math.soimeme.org/~arunram/Preprints/181212Cntfcns.pdf
\begin{definition}[Topology]
		$(E, T)$ is a topological space when $E$ is a set and $T$ is a \textit{topology} over $E$. That is, $T$ is a set of subsets of $E$, called open sets, such that:
		\begin{itemize}
				\item $\varnothing$ and $E$ are open: $\varnothing, E \in T$
				\item Unions (potentially infinite) of opens are open: Let $S \subseteq T$ be a set of opens. Then:
						$$\bigcup_{O \in S} O \in T$$
				\item Finite intersections of opens are open: Let $S = \{O_1, \dots, O_n\} \subseteq T$ be a finite set of opens. Then:
						$$\bigcap_{O \in S} O \in T$$
		\end{itemize}
\end{definition}

\begin{example}[Trivial topology]
		For any set $E$, we call $\{\varnothing, E\}$ the trivial topology of $E$. It is the smallest topology of $E$ by inclusion, that is, this topology is included in every other topology of $E$.
\end{example}

\begin{example}[Discrete topology]
		For any set $E$, we call $\mathcal{P}(E)$ the discrete topology of $E$. It is the biggest topology of $E$ by inclusion, that is, this topology includes every other topology of $E$.
\end{example}

\begin{definition}[Closed set]
		Let $(E, T)$ be a topological space. A subset $C \subseteq E$ is said to be closed when its complementary is open. We call $\closed = \{C \subseteq E \mid C^c \in T\}$ the set of closed sets. Alternatively, we could say that a subset $C \subseteq E$ is closed when it is the complementary of an open (in that case, it is the complementary of the open $C^c$). As such, we have $\closed = \{O^c \mid O \in T\}$.
\end{definition}

\begin{remark}
	Just as closed sets are sets whose complementary is open, open sets are sets whose complementary is closed, and therefore $T = \{O \subseteq E \mid O^c \in \closed\}$. Similarly, just as closed sets are the complementaries of open sets, open sets are the complementaries of closed sets, and therefore $T = \{C^c \mid C \in \closed\}$.
\end{remark}

\begin{definition}[Clopen sets]
		Let $(E, T)$ be a topological space. We say that $X \subseteq E$ is clopen when it is both open and closed.
\end{definition}

\begin{theorem}[$E$ and $\varnothing$ are clopen]
		Let $(E, T)$ be a topological space. Then, $\varnothing$ and $E$ are clopen.
\end{theorem}

\begin{proof}
		From the definition of topological space, we have that $\varnothing$ and $E$ are open. But, as they are both the complementary of each other, then they are the complementary of an open. They therefore are also closed.
\end{proof}

\begin{theorem}[Intersection of closed sets is closed]
		Let $(E, T)$ be a topological space, and let $S \subseteq \closed$ be a set of closed sets. Then $\bigcap_{C \in S} C$ is closed.
\end{theorem}

\begin{proof}
		To prove that $\bigcap_{C \in S} C$ is closed, we will prove that its complementary its open. From De Morgan's laws, we have:
				$$(\bigcap_{C \in S} C)^c = \bigcup_{C \in S} C^c$$
		And as all the $C \in S$ are closed, then the $C^c$ are open. We therefore have a union of opens, which is open.
\end{proof}

\begin{theorem}[Finite union of closed sets is closed]
		Let $(E, T)$ be a topological space, and let $S = \{C_1, \dots, C_n\} \subseteq \closed$ be a finite set of closed sets. Then $\bigcup_{C \in S} C$ is closed.
\end{theorem}

\begin{proof}
		From De Morgan's laws, the complementary of $\bigcup_{C \in S} C$ is:
				$$(\bigcup_{C \in S} C)^c = \bigcap_{C \in S} C^c$$
		which is a finite intersection of open, and is therefore open. Therefore, $\bigcup_{C \in S} C$ is closed.
\end{proof}

\subsection{Neighbourhood}

\begin{definition}[Neighbourhood of a set]
		Let $(E, T)$ be a topological space, and let $X \subseteq E$. A neighbourhood of $X$ is a subset $N$ that includes an open set which includes $X$. That is, $N$ is a neighbourhood of $X$ when:
				$$\exists O \in T, X \subseteq O \subseteq N$$
		We write $\mathcal{N}_E(X)$ for the set of all neighbourhoods of $X$ according to the topological space $(E, T)$. When there is no confusion about which topological space is considered, we just write $\mathcal{N}(X)$.
\end{definition}

\begin{definition}[Neighbourhood of a point]
		Let $(E, T)$ be a topological space, and let $x \in E$. $N \subseteq X$ is said to be a neighbourhood of $x$ when it is a neighbourhood of the singleton $\{x\}$. Equivalently, it is when there exists an open $O \in T$ such that $x \in O$ and $O \subseteq N$. We also use the notation $\mathcal{N}(x)$ for the neighbourhood of $x$.
\end{definition}

\begin{theorem}[All sets have a clopen neighbourhood] \label{thm:neighbourhood-existence}
		Let $(E, T)$ be a topological space, and let $X \subseteq E$. Then, $X$ has a clopen neighbourhood.
\end{theorem}

\begin{proof}
		This is trivial as $E$ is a clopen and is a neighbourhood of $X$ (since $X \subseteq E \subseteq E$ holds).
\end{proof}

\begin{theorem}[A set is open iff it is a neighbourhood of itself] \label{thm:open-eq-own-neighbourhood}
		Let $(E, T)$ be a topological space, and let $X \subseteq E$. $X$ is open if and only if it is a neighbourhood of itself.
\end{theorem}

\begin{proof}~
	\begin{itemize}
		\item Suppose that $X$ is open. As $X \subseteq X \subseteq X$ and that $X$ is open, then $X$ is a neighbourhood of itself.
		\item Suppose that $X$ is a neighbourhood of itself. Then, there exists an open $O \in T$ such that $X \subseteq O \subseteq X$, and therefore $X = O$, proving that $X$ is open.
	\end{itemize}
\end{proof}

\begin{theorem}[A set is open iff it is a neighbourhood of all its subsets] \label{thm:open-eq-neighbourhood}
		Let $(E, T)$ be a topological space, and let $X \subseteq E$. Then $X$ is open if and only if it is a neighbourhood of all its subsets.
\end{theorem}

\begin{proof}~
	\begin{itemize}
		\item Suppose that $X$ is open, and let $X' \subseteq X$. As $X' \subseteq X \subseteq X$, and that $X$ is open, then $X$ indeed is a neighbourhood of $X'$.
		\item Suppose that $X$ is a neighbourhood of all of its subsets. Then it is also a neighbourhood of itself, and from theorem (\ref{thm:open-eq-own-neighbourhood}), $X$ is open.
	\end{itemize}
\end{proof}

\subsection{Structures formed by neighbourhoods}

In this section, $(E, T)$ will correspond to a topological space, and $X \subseteq E$ will correspond to a subset of $E$.

\begin{definition}[Adhrerent point / Closure]
		$x \in E$ is said to be an adherent point of $X$ when every neighbourhood of $x$ intersect $X$:
				$$\forall N \in \mathcal{N}(x), X \cap N \neq \varnothing$$
		The closure $\overline{X}$ of $X$ is the set of all adhrent points of $X$.
\end{definition}

\begin{definition}[Interior point / Interior]
		$x \in E$ is said to be an interior point of $X$ when $X$ contains a neighbourhood of $x$:
				$$\exists N \in \mathcal{N}(X), N \subseteq X$$
		The interior $\inter{X}$ of $X$ is the set of all interior points of $X$.
\end{definition}

\begin{definition}[Boundary point / Boundary]
		$x \in E$ is said to be a boundary point of $X$ when every neighbourhood of $x$ contains at least one point in $X$ and one point not in $X$:
				$$\forall N \in \mathcal{N}(X), X \cap N \neq \varnothing \tand X^c \cap N \neq \varnothing$$
		The boundary $\partial X$ of $X$ is the set of all boundary points of $X$.
\end{definition}

\begin{definition}[Exterior point / Exterior]
		$x \in E$ is said to be an exterior point of $X$ when $x$ has a neighbourhood which doesn't intersect with $X$:
				$$\exists N \in \mathcal{N}(x), X \cap N = \varnothing$$
\end{definition}

\begin{definition}[Limit point / Derived set]
		$x \in E$ is said to be a limit point of $X$ when every neighbourhood of $x$, after removing $x$ from it, intersect $X$:
				$$\forall N \in \mathcal{N}(X), X \cap (N \setminus \{x\}) \neq \varnothing$$
		The derived set $X'$ of $X$ is the set of all limit points of $X$.
\end{definition}

\begin{definition}[Isolated point / Set of isolated points]
		$x \in E$ is said to be an isolated point of $X$ when $x$ has a neighbourhood whose intersection with $X$ is the singleton $\{x\}$:
				$$\exists N \in \mathcal{N}(X), X \cap N = \{x\}$$
		We note $X^i$ for the set of all isolated points of $X$.
\end{definition}

\begin{theorem}[Closure and interior are dual]
		We have:
				$$\inter{X} = \overline{X^c}^c$$
				$$\overline{X} = (\inter{(X^c)})^c$$
\end{theorem}

\begin{proof}
		Let's prove $\inter{X} = \overline{X^c}^c$:
		\begin{itemize}
				\item Let $x \in \inter{X}$. $X$ therefore contains a neighbourhood of $X$. Since this neighbourhood is contained in $X$, it cannot intersect $X^c$. Since $x$'s neighbourhoods therefore do not all intersect $X^c$, then $x \notin \overline{X^c}$, and therefore $x \in \overline{X^c}^c$.
				\item Let $x \in \overline{X^c}^c$. We therefore have that $x \notin \overline{X^c}$, which means that there is a neighbourhood of $x$ which doesn't intersect with $X^c$. This neighbourhood must therefore be contained in $X$, and we therefore have $x \in \inter{X}$
		\end{itemize}
		Let's now prove that $\overline{X} = (\inter{(X^c)})^c$:
		\begin{itemize}
				\item Let $x \in \overline{X}$. All of $x$'s neighbourhood therefore intersect $X$. There therefore is no neighbourhood of $x$ which don't intersect $X$. There therefore is no neighbourhood of $x$ which are contained in $X^c$. Therefore, $x \notin \inter{(X^c)}$, which means that $x \in (\inter{(X^c)})^c$
				\item Let $x \in (\inter{(X^c)})^c$. As $x \notin \inter{(X^c)}$, 
		\end{itemize}
\end{proof}

\subsection{Closure and interior}

\begin{notation}
		Let $(E, T$ be a topological space, and let $X \subseteq E$. We note $\closed(X) = \{C \in \closed \mid X \subseteq C\}$ the set of all closed sets which include $X$, and we note $T(X) = \{O \in T \mid O \subseteq X\}$ the set of all open sets which are included in $X$.
\end{notation}

\begin{definition}[Closure] %Adhérence
		Let $(E, T)$ be a topological space, and let $X \subseteq E$. The closure of $X$, noted $\overline{X}$ is the intersection of all closed sets that include $X$:
				$$\overline{X} = \bigcap_{C \in \closed(X)} C$$
\end{definition}

\begin{definition}[Interior]
		Let $(E, T)$ be a topological space, and let $X \subseteq E$. The interior of $X$, noted $\inter{X}$, is the union of all open sets that are included in $X$:
				$$\inter{X} = \bigcup_{O \in T(X)} O$$
\end{definition}

\begin{theorem}[Interior and closure are dual]
		Let $(E, T)$ be a topological space, and let $X \subseteq E$. Then, we have:
				$$\inter{X} = \overline{X^c}^c$$
				$$\overline{X} = (\inter{(X^c)})^c$$
\end{theorem}

\begin{proof}
		We have:
		\begin{align*}
				\inter{X} &= \bigcup_{O \in T(X)} O\\
						  &= \lr{\lr{\bigcup_{O \in \{O' \in T \mid O' \subseteq X\}} O}^c}^c\\
						  &= \lr{\bigcap_{O \in \{O' \in T \mid X^c \subseteq O'^c\}} O^c}^c\\
						  &= \lr{\bigcap_{C \in \{O^c \mid O \in T, X^c \subseteq O^c\}} C}^c\\
						  &= \lr{\bigcap_{C \in \closed(X^c)} C}^c \text{ (as $\{O^c \mid O \in T, X^c \subseteq O^c\} = \{C \in \closed \mid X^c \subseteq C\} = \closed(X^c)$)}\\
						  &= \overline{X^c}^c
		\end{align*}
		And therefore $\inter{X} = \overline{X^c}$. When taking $X^c$, it becomes $\inter{(X^c)} = \overline{(X^c)^c}^c = \overline{X}^c$, and therefore $\overline{X} = (\inter{(X^c)})^c$.
\end{proof}

\begin{theorem}[Characteristic property of closure] \label{thm:closure-charac}
		Let $(E, T)$ be a topological space, and let $X \subseteq E$. $\overline{X}$ is the smallest closed set which includes $X$. That is, $\overline{X}$ is the only set such that:
		\begin{itemize}
				\item $\overline{X} \in \closed$
				\item $X \subseteq \overline{X}$
				\item $\forall C \in \closed(X), \overline{X} \subseteq C$
		\end{itemize}
\end{theorem}

\begin{proof}
		Let's firstly prove that $\overline{X}$ respects these three properties:
		\begin{itemize}
				\item $\overline{X} \in \closed$: As $\overline{X}$ is an intersection of closed sets, then $\overline{X}$ is closed
				\item $X \subseteq \overline{X}$: As $\overline{X}$ is an intersection of sets which all include $X$, then $X \subseteq \overline{X}$. More formally, for all $C \in \closed(X) = \{C \in \closed \mid X \subseteq C\}$, since $X \subseteq C$, we have $C = X \cup C$. We therefore have:
						$$\overline{X} = \bigcap_{C \in \closed(X)} C = \bigcap_{C \in \closed(X)} (X \cup C) = X \cup (\bigcap_{C \in \closed(X)} C) = X \cup \overline{X}$$
				And therefore $X \subseteq X \cup \overline{X} = \overline{X}$
				\item $\forall C \in \closed(X), \overline{X} \subseteq C$: As $\overline{X}$ is the intersection of all elements of $\closed(X)$, then it must be included in all elements of $\closed(X)$. More formally, let $C' \in \closed(X)$. We need to prove that $\overline{X} \subseteq C$. As $C' \in \closed(X)$, then $\closed(X) = \closed(X) \cup \{C'\}$. Therefore:
						$$\overline{X} = \bigcap_{C \in \closed(X)} C = \bigcap_{C \in \closed(X) \cup \{C'\}} C = C' \cap \bigcap_{C \in \closed(X)} C = C' \cap \overline{X}$$
				And therefore $\overline{X} = C' \cap \overline{X} \subseteq C'$
		\end{itemize}
		We now need to prove that $\overline{X}$ is the only set respecting these three properties. Let's suppose there is another set $A$ such that:
		\begin{itemize}
				\item $A \in \closed$
				\item $X \subseteq A$
				\item $\forall C \in \closed(X), A \subseteq C$
		\end{itemize}
		As $\overline{X} \in \closed$ (first property of $\overline{X}$) and $X \subseteq \overline{X}$ (second property of $\overline{X}$), then $\overline{X} \in \closed(X)$, and therefore, from the third property of $A$, we have that $A \subseteq \overline{X}$. Similarly, as $A \in \closed$ and $X \subseteq A$, then $A \in \closed(X)$, and therefore $\overline{X} \subseteq A$. Therefore, $A = \overline{X}$, which proves that $\overline{X}$ is the unique set respecting these three properties.
\end{proof}

\begin{theorem}[Characteristic property of interior] \label{interior-charac}
		Let $(E, T)$ be a topological space, and let $X \subseteq E$. $\inter{X}$ is the biggest open set which is included in $X$. That is, $\inter{X}$ is the only set such that:
		\begin{itemize}
				\item $\inter{X} \in T$
				\item $\inter{X} \subseteq X$
				\item $\forall O \in T(X), O \subseteq \inter{X}$
		\end{itemize}
\end{theorem}

\begin{proof}
		As in the previous proof, let's prove that $\inter{X}$ respects these three properties:
		\begin{itemize}
				\item $\inter{X} \in T$: As $\inter{X}$ is a union of opens, it is open
				\item $\inter{X} \subseteq X$: As $\inter{X}$ is a union of sets which are all included in $X$, then $\inter{X}$ is also included in $X$
				\item $\forall O \in T(X), O \subseteq \inter{X}$: As $\inter{X}$ is the union of all elements of $T(X)$, then it must include all elements of $T(X)$.
		\end{itemize}
		Furthermore, if there were another set $A$ which were to have these properties, then since it would be open and would be included in $X$, it would be in $T(X)$, and therefore we would have $A \subseteq \overline{X}$. And furthermore, as $\overline{X} \in T(X)$, we would also have $\overline{X} \subseteq A$, and therefore these two sets would be the same.
\end{proof}

\begin{theorem}[Closed sets are their own closure] \label{thm:closed-eq-closure}
		Let $(E, T)$ be a topological space, and let $C \in \closed$ be a closed set. Then, $C = \overline{C}$
\end{theorem}

\begin{proof}
		From theorem (\ref{thm:closure-charac}), we just need to prove that:
		\begin{itemize}
				\item $C \in \closed$ (This is one of our hypothesis)
				\item $C \subseteq C$ (Trivial)
				\item $\forall C' \in \closed(C), C \subseteq C'$: This can be rewritten as $\forall C' \in \closed, C \subseteq C' \implies C \subseteq C'$, which is trivial.
		\end{itemize}
\end{proof}

\begin{theorem}[Open sets are their own interior] \label{thm:open-eq-interior}
		Let $(E, T)$ be a topological space, and let $O \in T$ be an open. Then $O = \inter{O}$
\end{theorem}

\begin{proof}
		From theorem (\ref{interior-charac}), we just need to prove that:
		\begin{itemize}
				\item $O \in T$ (This is one of our hypothesis
				\item $O \subseteq O$ (Trivial)
				\item $\forall O' \in T(O), O' \subseteq O$: This can be rewritten as $\forall O' \in T, O' \subseteq O \implies O' \subseteq O$, which is trivial.
		\end{itemize}
\end{proof}

\begin{corollary}[Closure of closure is closure]
		Let $(E, T)$ be a topological space, and let $X \subseteq E$. Then, $\overline{\overline{X}} = \overline{X}$ 
\end{corollary}

\begin{proof}
		As $\overline{X}$ is closed, then from corollary (\ref{thm:closed-eq-closure}), $\overline{X} = \overline{\overline{X}}$.
\end{proof}

\begin{corollary}[Interior of interior is interior]
		Let $(E, T)$ be a topological space, and let $X \subseteq E$. Then, $\inter{(\inter{X})} = \inter{X}$
\end{corollary}

\begin{proof}
		As $\inter{X}$ is open, then from corollary (\ref{thm:open-eq-interior}), $\inter{X} = \inter{(\inter{X})}$.
\end{proof}

\begin{theorem}[Union of closures is closure of the union]
		Let $(X, T)$ be a topological space, and let $X, Y \subseteq E$. Then, $\overline{X} \cup \overline{Y} = \overline{X \cup Y}$
\end{theorem}

\begin{proof}
		Since $\overline{X \cup Y}$ is closed and includes $X$ (since $X \subseteq X \cup Y \subseteq \overline{X \cup Y}$), then it includes $\overline{X}$. Similarly, since $\overline{X \cup Y}$ is closed and includes $Y$ (since $Y \subseteq X \cup Y \subseteq \overline{X \cup Y}$), then it includes $\overline{Y}$. Therefore:
				$$\overline{X} \cup \overline{Y} \subseteq \overline{X \cup Y}$$
		As $X \subseteq \overline{X} \subseteq \overline{X} \cup \overline{Y}$ and $Y \subseteq \overline{Y} \subseteq \overline{X} \cup \overline{Y}$, then $X \cup Y \subseteq \overline{X} \cup \overline{Y}$. Since $\overline{X} \cup \overline{Y}$ is closed (since it is a finite union of closed sets) and includes $X \cup Y$, then it includes $\overline{X \cup Y}$. Therefore:
				$$\overline{X \cup Y} \subseteq \overline{X} \cup \overline{Y}$$
		Which proves that $\overline{X} \cup \overline{Y} = \overline{X \cup Y}$.
\end{proof}

\begin{corollary}[Intersection of interiors is interior of the intersection]
		Let $(X, T)$ be a topological space, and let $X, Y \subseteq E$. Then, $\inter{X} \cap \inter{Y} = \inter{(X \cap Y)}$
\end{corollary}

\begin{proof}
		$$\inter{X} \cap \inter{Y} = \overline{X^c}^c \cap \overline{Y^c}^c = (\overline{X^c} \cup \overline{Y^c})^c = \overline{X^c \cup Y^c}^c = \overline{(X \cap Y)^c}^c = \inter{(X \cap Y)}$$
\end{proof}

\begin{remark}
		The intersection of closures is not necessarily the closure of the intersection, and the union of interiors is not necessarily the interior of the union.
\end{remark}

\subsection{Adherent and interior points}

\begin{definition}[Adherent point] %different than limit point (explained on the english wikipedia)
		Let $(E, T)$ be a topological space, and let $X \subseteq E$. Then $x \in E$ is said to be an adherent point of $X$ (or a closure point of $X$, or a contact point of $X$) when every neighbourhood of $x$ has a non-null intersection with $X$. That is:
				$$\forall N \in \mathcal{N}(x), N \cap X \neq \varnothing$$
\end{definition}

\begin{definition}[Interior point]
		Let $(E, T)$ be a topological space, and let $X \subseteq E$. Then $x \in E$ is said to be an interior point of $X$ when some neighbourhood of $x$ is included in $X$:
				$$\exists N \in \mathcal{N}(X), N \subseteq X$$
		Equivalently, $x \in E$ is an interior point of $X$ when $X$ is a neighbourhood of $x$.
\end{definition}

\begin{theorem}[Characteristic property of interior point]
		Let $(E, T)$ be a topologial space, and let $X \subseteq E$. Then $x \in E$ is an interior point of $X$ if and only if $X$ is a neighbourhood of $x$.
\end{theorem}

\begin{proof}
		Let's prove the equivalence:
		\begin{itemize}
				\item If $x$ is an interior point of $x$, then there exists a neighbourhood $N \in \mathcal{N}(x)$ of $x$ such that $N \subseteq X$. From the definition of neighbourhood, there exists an open $O \in T$ such that $x \in O \subseteq N$, and therefore we have $x \in O \subseteq X$, which means that $X$ is a neighbourhood of $x$.
				\item If $X$ is a neighbourhood of $x$, then there trivially exists a neighbourhood $N \in \mathcal{N}(x)$ of $x$ such that $N \subseteq X$, as we can take $N = X$.
		\end{itemize}
\end{proof}

\begin{theorem}[Characteristic property of adherent point] \label{thm:adherent-carac}
		The above definition also works if we consider only the open neighbourhoods instead of every neighbourhood. That is, let $(E, T)$ be a topological space, and let $X \subseteq E$. Then $x \in E$ is an adherent point of $X$ if and only if:
				$$\forall O \in \mathcal{N}(x) \cap T, O \cap X \neq \varnothing$$
\end{theorem}

\begin{proof}
		Let's prove the implication and then the reciprocal.
		\begin{itemize}
				\item If $x \in E$ is an adherent point of $X$, then as all neighbourhoods of $x$ have a non-null intersection with $X$, then this is also the case for all open neighbourhoods of $X$.
				\item Let $x \in E$ such that $\forall O \in \mathcal{N}(x) \cap T, O \cap X \neq \varnothing$, and let $N \in \mathcal{N}(x)$ be a neighbourhood of $x$. To prove that $x$ is an adherent point of $X$, we just need to prove that $N \cap X \neq \varnothing$. As $N$ is a neighbourhood of $x$, we have from the definition of neighbourhood that there exists an open $O \in T$ such that $x \in O$ and $O \subseteq N$. As $O$ is open, we have from theorem (\ref{thm:open-eq-neighbourhood}) that it is a neighbourhood for all its points, including $x$. Since $O$ is a neighbourhood of $x$ and is an open, then $O \in \mathcal{N}(x) \cap T$, and therfore from the hypothesis we have that $O \cap X \neq \varnothing$. As $O \subseteq N \neq \varnothing$, we therefore have $N \cap X \neq \varnothing$, which proves that $x$ is an adherent poit of $X$.
		\end{itemize}
\end{proof}

\begin{theorem}[Adherent points are the points in the closure]
		Let $(E, T)$ be a topological space, and let $X \in E$. Then, $x \in E$ is an adherent point of $X$ if and only if $x \in \overline{X}$
\end{theorem}

\begin{proof}
		Let's prove the implication and then the reciprocal:
		\begin{itemize}
				\item Suppose that $x \in E$ is an adherent point of $X$ but that $x \notin \overline{X}$. From theorem (\ref{thm:neighbourhood-existence}), there exists an open neighbourhood $O \in \mathcal{N}(x) \cap T$ of $x$. Let $O'$ such that:
						$$O' = O \cap \overline{X}^c = O \setminus \overline{X}$$
				As $O'$ is the intersection of two opens, it is an open. We therefore have from theorem (\ref{thm:open-eq-neighbourhood}) that it is a neighbourhood of all its points. And, as $x \in O$ but that $x \notin \overline{X}$, we have that $x \in O \setminus \overline{X} = O'$. Therefore, $O'$ is a neighbourhood of $x$. But, we have:
						$$O' \cap \overline{X} = (O \setminus \overline{X}) \cap \overline{X} = \varnothing$$
				which contradicts the fact that $x$ is an adherent pointf :  of $X$.
				\item Suppose that $x \in \overline{X}$ but that $x$ isn't an adherent point of $X$. Then, as $x$ isn't an adherent point of $X$, there exists an open neighbourhood $O \in \mathcal{N}(x) \cap T$ of $x$ such that $O \cap X = \varnothing$. Now, let $X'$ such that:
						$$\overline{X}' = \overline{X} \cap O^c = \overline{X} \setminus O$$
				As $\overline{X}'$ is an intersection of two closed sets, it is closed. Furthermore, as $O \cap X = \varnothing$, then:
						$$X = X \setminus O \subseteq \overline{X} \setminus O = \overline{X}'$$
				As such, $\overline{X}'$ is a closed set which includes $X$. It therefore is in $S = \{C \in \mathcal{C} \mid X \subseteq C\}$, and therefore we have $S = S \cup \{\overline{X}'\}$, which implies that:
						$$\overline{X} = \bigcap_{C \in S} C = \bigcap_{C \in S \cup \{\overline{X}'\}} C = \overline{X}' \cap \bigcap_{C \in S} C = \overline{X}' \cap \overline{X} \subseteq \overline{X}'$$
				And, as $x \in \overline{X}$, then $x \in \overline{X'}$. But, as $x \in O$, then $x \notin \overline{X} \setminus O = \overline{X'}$, contradicting the fact that $x \in \overline{X'}$.
		\end{itemize}
\end{proof}

\begin{remark}
		The previous proof was non-constructive.
\end{remark}





\begin{definition}[Limit point]
		Let $(X, T)$ be a topological space, and let $X \subseteq E$. We say that $x \in E$ is
\end{definition}




\begin{notation}[Image of a set]
		Let $X$ and $Y$ be sets, let $f : X \rightarrow Y$, and let $A \subseteq X$. We write $f(A)$ for $\{f(a) \mid a \in A\}$. This notation isn't problematic, unless the set $A$ is itself in $X$, in which case we cannot know if $f(A)$ just means the image of $A$ or the set of images of all points of $A$.
\end{notation}

\begin{definition}[Limit]
		Let $(X, T_X)$ and $(Y, T_Y)$ be two topological spaces, and let $A \subseteq E$. We say that the limit of $f : A \rightarrow F$ at point $a \in \overline{A}$ is $l \in F$ when:
				$$\forall N \in \mathcal{N}_Y(y), \exists M \in \mathcal{N}_X(a), f(M \cap A) \subseteq N$$

		Let $(X, T_X)$ and $(Y, T_Y)$ be topological spaces. We say that the limit of $f : X \rightarrow Y$, at point $a \in X$ is $y \in Y$ when:
				$$\forall N \in \mathcal{N}_Y(y), \exists M \in \mathcal{N}_X(a), f(M) \subseteq N$$

\end{definition}



\end{document}
