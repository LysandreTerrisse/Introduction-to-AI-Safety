\documentclass{article}
\usepackage{amsmath}
\usepackage{amssymb}
\usepackage{amsthm}
\newtheorem{theorem}{Theorem}[section]
\newtheorem{lemma}[theorem]{Lemma}
\newtheorem{corollary}[theorem]{Corollary}
\renewcommand*{\proofname}{Proof}
\theoremstyle{definition}
\newtheorem{definition}{Definition}[section]
\theoremstyle{remark}
\newtheorem*{remark}{Remark}
\theoremstyle{example}
\newtheorem{example}{Example}[section]
\DeclareMathOperator*{\argmax}{arg\,max}
\usepackage{fullpage}
\usepackage{outlines}
\allowdisplaybreaks

\title{Measure Theory}
\author{Lysandre Terrisse}

\begin{document}
\maketitle

\section{Measure theory}

\begin{definition}[Measurable space and $\sigma$-algebra]
    Let $X$ be a set. A \textit{measurable space} is a couple $(X, \mathcal{A})$ where $\mathcal{A}$ is a \textit{$\sigma$-algebra} of $X$. That is, $\mathcal{A}$ is a set of subsets of $X$, called \textit{measurable sets}, such that:
    \begin{itemize}
        \item $X$ is a measurable set:
            $$X \in \mathcal{A}$$
        \item Complements of measurable sets are measurable:
            $$\forall A \in \mathcal{A}, A^c \in \mathcal{A}$$
        \item Countable unions of measurable sets are measurable: For all countable family $(A_i)_{i \in \mathbb{N}}$ over measurable sets:
            $$\bigcup_{i \in \mathbb{N}} A_i \in \mathcal{A}$$
    \end{itemize}
\end{definition}

\begin{theorem}[Countable intersections of measurable sets are measurable]
    Let $(X, \mathcal{A})$ be a measurable space, and let $(A_i)_{i \in \mathbb{N}}$ be a countable family over measurable sets. Then:
            $$\bigcap_{i \in \mathbb{N}} A_i \in \mathcal{A}$$
\end{theorem}

\begin{proof}
    As $\bigcup_{i \in \mathbb{N}} A_i \in \mathcal{A}$, then $\left(\bigcup_{i \in \mathbb{N}} A_i\right)^c = \bigcap_{i \in \mathbb{N}} A_i \in \mathcal{A}$.
\end{proof}

\begin{definition}[Trivial and full $\sigma$-algebra]
    Let $X$ be a set. The \textit{trivial $\sigma$-algebra} is $\{\varnothing, X\}$, and the \textit{full $\sigma$-algebra} is $\mathcal{P}(X)$. They indeed are $\sigma$-algebras, since $X$ belongs to them, and that they are stable over complements and countable unions.
\end{definition}

\begin{theorem}[$\sigma$-algebra are between the trivial and full $\sigma$-algebras]
    For all measurable space $(X, \mathcal{A})$, we have:
        $$\{\varnothing, X\} \subseteq \mathcal{A} \subseteq \mathcal{P}(X)$$
\end{theorem}

\begin{proof}
    Since $X \in \mathcal{A}$, and that $\mathcal{A}$ is closed under complement, then $X^c = X \setminus X = \varnothing \in \mathcal{A}$. Therefore, $\{\varnothing, X\} \subseteq \mathcal{A}$. Furthermore, from the definition of $\sigma$-algebra, $\mathcal{A} \subseteq \mathcal{P}(X)$. 
\end{proof}

\begin{remark}
    This theorem implies that the trivial and the full $\sigma$-algebras are respectively the smallest and biggest $\sigma$-algebras. By \textit{smallest}, we mean that all other $\sigma$-algebras of $X$ contain it, and by \textit{biggest}, we mean that all other $\sigma$-algebras of $X$ are contained in it.
\end{remark}

\begin{theorem}[Intersections of $\sigma$-algebras are $\sigma$-algebras]
    Let $X$ be a set, and let $(\mathcal{A}_i)_{i \in I}$ be a (potentially uncountable) family of $\sigma$-algebras of $X$. Then $\bigcap_{i \in I} \mathcal{A}_i$ is a $\sigma$-algebra.
\end{theorem}

\begin{proof}~
    \begin{itemize}
        \item As we have $X \in \mathcal{A}_i$ for all $i \in I$, then we have $X \in \bigcup_{i \in I} \mathcal{A}_i$.
        \item If $A \in \bigcup_{i \in I} \mathcal{A}_i$, then $A \in \mathcal{A}_i$ for all $i \in I$, and therefore $A^c \in \mathcal{A}_i$ for all $i \in I$, meaning that $A^c \in \bigcup_{i \in I} \mathcal{A}_i$.
        \item Let $(A_j)_{j \in J}$ be a countable family over $\bigcup_{i \in I} \mathcal{A}_i$. This family is also over $\mathcal{A}_i$ for all $i \in I$. Therefore, $\bigcup_{j \in J} A_j \in \mathcal{A}_i$ for all $i \in I$, meaning that $\bigcup_{j \in J} A_j \in \bigcup_{i \in I} \mathcal{A}_i$.
    \end{itemize}
\end{proof}

\begin{remark}
    Unions of $\sigma$-algebras are not necessarily $\sigma$-algebras.
\end{remark}

\begin{definition}[Generated $\sigma$-algebra]
    Let $X$ be a set, and let $Y \subseteq \mathcal{P}(X)$. The $\sigma$-algebra of $X$ generated by $Y$, denoted $\sigma(Y)$, is the smallest $\sigma$-algebra of $X$ containing all the sets of $Y$. That is, $\sigma(Y)$ is defined as the only set such that:
    \begin{itemize}
        \item $\sigma(Y)$ is a $\sigma$-algebra of $X$
        \item $Y \subseteq \sigma(Y)$
        \item For all $\sigma$-algebra $\mathcal{A}$ of $X$ such that $Y \subseteq \mathcal{A}$, we have $\sigma(Y) \subseteq \mathcal{A}$.
    \end{itemize}
    The uniqueness of such $\sigma$-algebra can be proven directly: for any two sets respecting these properties, they would from the third property include each other, and would therefore be equal. The existence such $\sigma$-algebra is proven in the next theorem.
\end{definition}

\begin{theorem}[Characteristic property of $\sigma$-algebra]
    Let $X$ be a set, let $Y \subseteq \mathcal{P}(X)$, and let $Z = \{\mathcal{A} \mid \text{$Y \subseteq \mathcal{A}$ and $\mathcal{A}$ is a $\sigma$-algebra of $X$}\}$. Then we have:
        $$\sigma(Y) = \bigcap_{\mathcal{A} \in Z} \mathcal{A}$$
\end{theorem}

\begin{proof}
    Firstly, $\bigcap_{\mathcal{A} \in Z} \mathcal{A}$ is an intersection of $\sigma$-algebras of $X$, and is therefore a $\sigma$-algebra of $X$. Secondly, as $Y \subseteq \mathcal{A}$ for all $\mathcal{A} \in Z$, we have that $Y \subseteq \bigcap_{\mathcal{A} \in Z} \mathcal{A}$. Thirdly, $\bigcap_{\mathcal{A} \in Z} \mathcal{A} \subseteq \mathcal{A}'$ for all $\mathcal{A}' \in Z$. Therefore, $\bigcap_{\mathcal{A} \in Z} \mathcal{A}$ respects all the conditions of the $\sigma$-algebra, meaning that $\sigma(Y) = \bigcap_{\mathcal{A} \in Z} \mathcal{A}$.
\end{proof}

\end{document}




%https://www.math.lsu.edu/~sengupta/7312s02/sigmaalg.pdf
%https://moodle.umontpellier.fr/pluginfile.php/2373208/mod_resource/content/12/HAX503X-poly.pdf








